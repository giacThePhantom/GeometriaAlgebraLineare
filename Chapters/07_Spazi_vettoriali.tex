\chapter{Spazi vettoriali}
\section{Gruppo}
Un gruppo G \`e un insieme dotato di un'operazione $*$, cio\`e di una funzione $*:GxG\rightarrow G$ che deve essere associativa, deve esistere un elemento neutro $e$, ogni
elemento ammette un elemento simmetrico. Se l'operazione \`e commutativa il gruppo si chiama commutativo o abeliano. Un esempio di gruppo: $(\mathbb{Z},+)$. 
\section{Campo}
Un campo K \`e un insieme dotato di due operazioni, $+$ e $*$ tali che rispetto alla somma K sia un gruppo abeliano e denotato con zero l'elemento neutro della somma, $(K
\backslash\{0\},*)$ \`e un gruppo abeliano. Le due operazioni devono essere distributive.
\section{Spazio vettoriale}
Uno spazio vettoriale sul campo K \`e un insieme V, nel quale \`e definita un'operazione $+:VxV\rightarrow V$, e $(V,+)$  \`e  un gruppo commutativo e un'operazione 
$KxV\rightarrow V$, detta prodotto per uno scalare tale che (Vedi propriet\`a prodotto per uno scalare tra due vettori). Gli elementi di V, denotati con lettere sottolineate 
o in grassetto sono detti Vettori. L'elemento neutro rispetto alla somma si dice vettore nullo: \underline{0}. Verranno considerati $K=R$, spazi vettoriali reali. 
\section{$\mathbb{R}[x]$}
$\mathbb{R}[x]$, i polinomi a coefficienti reali nella variabile x. La somma \`e l'usuale somma di polinomi, cos\`i come il prodotto per uno scalare.
$\mathbb{R}_d[x]$ polinomi a coefficienti reali di grado $\le d$.
\section{Sottospazio}
Un sottospazio di V \`e un sottoinsieme non vuoto U tale che $u_1+u_2\in U \forall u_1,u_2\in U$ e $\lambda u\in U\forall u\in U \forall\lambda\in\mathbb{R}$, cio\`e chiuso 
rispetto alle operazioni di V, equivalentemente con le combinazioni lineari. Se U \`e un sottospazio allora contiene il vettore nullo. Se U \`e un sottospazio non ridotto al
vettore nullo allora ha infiniti elementi.\\
Una combinazione lineare di vettori \`e un vettore formato da $\sum\limits_{i=1}^k \lambda_i v_i$. Il sottospazio generato da un insieme di vettori, denotato come sopra \`e 
un insieme delle combinazioni lineari dei vettori che lo generano.\\
V si dice finitamente generato se esistono dei vettori in V tali che lo spazio generato dai vettori sia V e i vettori si chiamano sistema di generatori per V.\\
Lo spazio dei polinomi non \`e finitamente generato.\\
V spazio vettoriale \`e dipendente se esiste una combinazione lineare con coefficienti non tutti nulli che faccia il vettore nullo, altrimenti si dice indipendente.\\
Dove c'\`e il vettore nullo il sistema \`e dipendente, e se \`e dipendente c'\`e un elemento che si scrive come combinazione lineare dei rimanenti. \\
Una base \`e un sistema di generatori linearmente indipendente.