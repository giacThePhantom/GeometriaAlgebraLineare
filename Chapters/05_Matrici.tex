\chapter{Matrici}
Una matrice di ordine mxn \`e una tabella con m righe e n colonne, nel nostro caso contenente numeri reali e si nota come: $M_{mxn} (\mathbf{R})$ e i suoi elementi
$A=[a_{ij}]$, con i indice riga e j indice colonna.
\subsubsection{Matrice quadrata}
Una matrice si dice quadrata se il numero di colonne \`e uguale a quello delle righe.
\subsubsection{Matrice identica}
 Una matrice quadrata di ordine m si dice identica se i suoi elementi: 
$I_m=[\delta_{[ij]}]$ se $\delta_i=1$ se i=j, $\delta_i=0$, se $i\neq j$.
\subsubsection{Trasposta della matrice}
Si chiama trasposta della matrice $A\in M_{mxn}(\mathbb{R})$ la matrice $A^t$ in cui le colonne sono le righe di A e le righe sono le colonne di A.
\subsubsection{Matrice simmetrica}
Se la matrice \`e quadrata e uguale alla sua trasposta allora si dice simmetrica. 
\section{Operazioni tra matrici}
\subsection{Somma tra matrici}
La somma di matrici dello stesso ordine si ottiene sommando i termini delle due matrici nella stessa posizione. \`E commutativa e associativa, la matrice nulla \`e 
l'elemento neutro e la matrice opposta \`e la matrice i cui elementi sono opposti all'altra matrice. 
\subsection{Prodotto di una matrice per uno scalare}
Gli elementi della matrice vengono tutti moltiplicati per il numero.
\subsection{Prodotto tra matrici}
$A\in M_{mxn} (\mathbb{R}), B\in M_{nxp} (\mathbb{R})$, \`e possibile definire il prodotto solo se il numero di colonne di A \`e uguale al numero di righe di B e A si dice 
conformabile a sinistra a B o B conformabile a destra a A. $AB\in M_{mxp} (\mathbb{R})$, l'elemento di indice ij di AB si ottiene moltiplicando termine a termine la riga i 
di A e la colonna j di B e sommando i vari prodotti: $c_{ik}=\sum\limits_{j=1}^n a_{ij}b_{jk}$. Tra le matrici quadrate il prodotto \`e sempre definito e genera una matrice 
dello stesso ordine. Non vale la propriet\`a commutativa, l'elemento neutro \`e la matrice identica. 
\subsection{Altra forma di operazioni sulle righe delle matrici}
Data una matrice $A\in M_{mxn} (\mathbb{R})$ ogni operazione elementare sulle righe pu\`o essere realizzata moltiplicando a sinistra per la matrice che si ottiene dalla 
matrice identica effettuando la stessa operazione elementare.
\section{La matrice inversa}
Data $A\in M_{m} (\mathbb{R}),\exists B\in M_{m} (\mathbb{R}): AB=BA=I_m$. Se B esiste allora si dice matrice inversa di A, $B=A^{-1}$. A \`e detta matrice invertibile e vale
solo per matrici quadrate. La matrice nulla non \`e invertibile e, pi\`u in generale, una matrice non \`e invertibile se possiede una riga o una colonna di zeri. L'inversa 
dell'identit\`a \`e l'identit\`a stessa: $I_m^{-1}=I_m$
\subsection{Unicit\`a della matrice inversa}
Supponiamo che $A\in M_{m} (\mathbb{R})$ sia invertibile e che possieda due matrici inverse $B,B'$:\\
$AB=BA=I\wedge AB'=B'A=I$\\
$B=BI=B(AB')$\\
$B=(BA)B'=IB'=B'$\\
$B=B'$, che \`e un assurdo.
\subsection{Prodotto di matrici invertibili}
Se A e B sono invertibili allora \`e invertibile anche AB e la sua inversa \`e $B^{-1}A^{-1}$\\
$(AB)(B^{-1}A^{-1})=A(BB^{-1})A^{-1}=AIA^{-1}=AA^{-1}=I$
\subsection{Trovare la matrice inversa}
Dato $A\in M_{m} (\mathbb{R})$, rref(A) si ottiene moltiplicando a sinistra A per un'opportuna matrice invertibile quadrata di ordine m. Pertanto $\exists P$ invertibile$: 
PA=rref(A)$. Per stabilire se A \`e invertibile devo ottenere $rref(A)$: se  $rk(rref(A))$ \`e massimo $rref(A)=I_m$, allora rref(A) \`e invertibile. $rk(rref(A))=n\Rightarrow
\exists n$ pivot, unico elemento diverso da zero nella loro colonna, in ogni riga \`e presente un pivot. Una matrice \`e invertibile solo se ha rango n massimo.\\
$\exists P\in M_n (\mathbb{R})$ invertibile: $PA=rref(A)$. Se A \`e invertibile allora PA \`e invertibile, ovvero rref(A) ha rango n e A ha rango n, ovvero rref(A)=I, 
$P=A^{-1}$. Per trovare la matrice inversa scrivo la matrice $(A|I_n)$, riduco a scalini la parte di A, verifico sia invertibile e la riduco all'identit\`a, in questo modo
avr\`o ottenuto la matrice $(I_n|A^{-1})$
\section{Matrici e sistemi lineari}
Un sistema lineare pu\`o essere scritto nella forma A\underline{x}=\underline{b}, dove A \`e la matrice rappresentante i coefficienti, \underline{x} quella rappresentante le
incognite e \underline{b} quella rappresentante i termini noti.
\begin{itemize}
\item  Se \underline{b}=0 il sistema \`e chiuso rispetto alle operazioni di somma e prodotto per uno scalare;
\item Se \underline{b}$\neq 0$ e $x_0$ \`e soluzione del sistema, tutte le soluzioni del sistema sono nella forma $x_0+v$, dove v \`e soluzione del sistema omogeneo associato.
\item Se la matrice quadrata A \`e invertibile il sistema ha un'unica soluzione nella forma $\underline{x}=A^{-1}\underline{b}$
\end{itemize}