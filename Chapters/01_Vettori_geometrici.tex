\chapter{Vettori geometrici nello spazio}
Un segmento orientato nello spazio \`e un segmento ove si \`e scelto un punto iniziale e un punto finale. Due segmenti orientati sono equivalenti se giacciono su rette
parallele con la stessa lunghezza e la stessa orientazione. Un vettore geometrico \`e una classe di equivalenza di segmenti orientati. La classe di equivalenza di segmenti 
orientati AA si dice vettore nullo ($\underline{0}$). Un vettore geometrico si nota come: $\underline{v}$. Un vettore geometrico non nullo \`e individuato da:
\begin{itemize}
\item direzione
\item modulo 
\item verso
\end{itemize}
\subsubsection{Vettori particolari}
\begin{itemize}
\item L'opposto di un vettore \`e un vettore con stesso modulo e stessa direzione ma verso opposto. 
\item Un versore \`e un vettore con modulo 1.
\end{itemize}
\section{Operazioni tra vettori}
\subsection{Somma di vettori}
La somma di vettori geometrici($\underline{v}+\underline{w}$) associa a due vettori un terzo vettore risultante: applico il primo vettore nel punto A, il secondo nella punta 
del primo e la somma \`e il 
vettore che parte dalla coda del primo e arriva alla punta del secondo. Il vettore nullo \`e l'elemento neutro della somma vettoriale. La somma di vettori geometrici \`e 
commutativa e associativa.
\subsection{Prodotto per uno scalare}
Il prodotto per uno scalare ($\lambda\underline{v}$) \`e un'operazione che associa un vettore geometrico ad uno scalare (un numero). Il vettore risultante \`e un vettore con 
verso dipendente dal segno dello scalare (stesso se positivo e opposto se negativo) e stessa direzione del vettore di partenza, ma il cui modulo \`e il prodotto tra il modulo 
del vettore iniziale e dello scalare. Se lo scalare \`e 0 si ottiene il vettore nullo.
\subsection{Normalizzazione di un vettore}
La normalizzazione di un vettore ($\frac{1}{|\underline{v}|}\underline{v}$) \`e il procedimento per cui si ottiene il versore con stessa direzione e verso di un qualsiasi 
vettore, attraverso un prodotto per uno scalare con l'inverso del modulo del vettore stesso. 
\subsection{Prodotto scalare tra vettori}
Prodotto scalare di vettori geometrici ($\underline{vw}=|\underline{v}||\underline{w}|\cos\theta$): da due vettori restituisce un numero: il prodotto scalare \`e il prodotto 
dei moduli dei vettori applicati nello stesso punto per il coseno dell'angolo compreso tra essi. \`E nullo quando uno dei due moduli \`e nullo, o quando l'angolo che formano 
tra di loro vale 90 gradi. \`E commutativo e distributivo
rispetto alla somma.
\subsection{Proiezione di un vettore su un altro}
Proiezione di un vettore su un altro: nel caso in cui voglia ottenere la proiezione di \underline{v} sul versore \underline{w} considero un vettore di direzione $\underline{w}
$ e modulo $\underline{v}\underline{w}$ e con lo stesso verso se l'angolo $\theta$ \`e acuto, verso opposto se \`e ottuso. Se \underline{w} non \`e versore lo devo 
normalizzare.